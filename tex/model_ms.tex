% XeLaTeX template for scientific papers using biblatex
\documentclass[11pt,a4paper]{article}

\usepackage[margin=2.5cm]{geometry}

\usepackage{titlesec}
\titleformat*{\section}{\Large\bfseries}
\titleformat*{\subsection}{\bfseries}

% left align title author blocks
\usepackage{titling}
\pretitle{\begin{flushleft} \Large}
\posttitle{\end{flushleft}}
\preauthor{\begin{flushleft}}
\postauthor{\end{flushleft}}

% suppress date
\predate{}
\postdate{}
\date{}

% left aligned abstract and title
\renewenvironment{abstract}
  {{\bfseries\noindent{\abstractname}\par\noindent}}
  


% fonts
\usepackage{fontspec}
\setmainfont{Adobe Text Pro}
% \setmainfont{Georgia}  % another nice option
\setsansfont{Source Sans Pro}

\usepackage{amsmath}

\usepackage{hyperref}
\usepackage{xcolor}
\hypersetup{
    colorlinks,
    linkcolor={red!50!black},
    citecolor={blue!50!black},
    urlcolor={blue!80!black}
}


% biblatex
\usepackage[style=authoryear,backend=biber, uniquelist=false, minbibnames=5, maxbibnames=5, 
		isbn=false,url=false, hyperref=true]{biblatex}  
\addbibresource{biblio.bib}	% fill in the name of the .bib file
\renewbibmacro{in:}{}
\DeclareFieldFormat[article]{title}{#1}  %% removes quotes from article titles

\usepackage{authblk} 

\usepackage{graphicx}


\usepackage[nolist,nohyperlinks]{acronym}


%%%% OPTIONS TO ENABLE FOR A FINAL (SUBMISSION) DRAFT
% \usepackage{setspace}
% \doublespacing
% \usepackage{lineno}
% \renewcommand\linenumberfont{\normalfont\small\sffamily}
% \linenumbers
% \usepackage[nomarkers]{endfloat} %% Enable this to have all figures and tables appear at the end of the ms




%% ----------------------------------
%
%     Title and authorship information
%
%% ----------------------------------

\title{Article title}
\author[1,*]{Matthew V. Talluto (matthew.talluto@uibk.ac.at)}
\author[2]{Someone else ()}
\author[1,3]{Gabriel A. Singer (gabriel.singer@uibk.ac.at)}
\affil[1]{Department of Ecology, University of Innsbruck, Innsbruck, Austria}
\affil[2]{Some other affiliation}
\affil[3]{Leibniz-Institute of Freshwater Ecology and Inland Fisheries (IGB), Berlin, Germany}
\affil[*]{Author for correspondence. Address: 
	\protect\\ \hspace{3em} University of Innsbruck 
	\protect\\ \hspace{3em} Department of Ecology 
	\protect\\ \hspace{3em} Technikerstrasse 25 
	\protect\\ \hspace{3em} A-6020 Innsbruck, Austria
	\protect\\ \hspace{3em} tel: +43 (0)512 507-51738}

\begin{document}

\begin{acronym}
	\newacro{fme}[FME]{fluvial meta-ecosystem}
\end{acronym}


\begin{titlepage}
	\maketitle
	\begin{flushleft}
		\textbf{Paper type:} \\
		\textbf{Short title:} \\
		\textbf{Keywords:} fluvial meta-ecosystems, biodiversity, ecosystem functioning, river networks
	\end{flushleft}

	\begin{abstract}
		Abstract here
	\end{abstract}
\end{titlepage}

\section{Introduction}

- Blah blah metaecosystems, explicit modelling, rivers

- Paragraph 2

- Paragraph 3

Much work connecting diversity with ecosystem functioning is conducted at the scale of local ecosystems.
Meta-ecosystems in general, and in particular \acp{fme}, however, connect multiple ecosystems across much larger scales (e.g., an entire fluvial network).
Biodiversity-ecosystem functioning relationships are likely quite scale dependent \autocite{Gonzalez2020}, and thus models that integrate the relevant processes from local to meta-ecosystem scales are needed to better understand both local and regional dynamics.

Fluvial ecosystems present unique problems for modelling.
\begin{itemize}
\item strong separation between river and surrounding matrix
\item The hierarchical branching structure that dictates the movement of material and species can produce both quantitative and qualitative changes in the predictions of models when compared with terrestrial systems. For example, branching structure may promote ecosystem stability and act as a buffer against environmental change \autocite{Terui2018}.
\item others?
\end{itemize} 




Here, we develop a model for \acp{fme} that incorporates the movement and distribution of non-biological material as well as the distribution of species in a biological community.
We construct this model by coupling two commonly-used model types in fluvial ecosystem and community ecology: reaction-transport models (for the non-biological component), and meta-community models (for species).
By linking these two models, we allow for feedbacks between the biological community and resources.
These feedbacks are essential to enable a more mechanistic understanding of the connections between the processes underlying community assembly, resulting biodiversity patterns, and ecosystem function.
Moreover, the spatial structure of the river network allows us to test emergent properties related to the structure of the network.
For example, changes to network structure via damming can be evaluated in-silico, or theoretical predictions such as the relationship between branching complexity and ecosystem stability \autocite{Terui2018} can be evaluated.

\section{Materials \& Methods}

\subsection{Model Description}

Our conceptual starting point for modelling the biological community is metapopulation theory \autocite{Levins1969}.
The classic metapopulation model tracks the number of occupied patches $p$ in a landscape composed of $h$ available patches as a function of the rates of colonisation and extinction of local populations within those patches:

\begin{equation}
	\frac{dp}{dt} = cp \left( h - p \right) - pm \label{eq:levins}
\end{equation}

Occupied patches ($p$) experience extinction according to the extinction rate $m$, while unoccupied patches ($h-p$) are colonised according to the colonisation rate $c$. 
The additional $p$ in the colonisation term takes dispersal into account; as more patches are occupied in the metapopulation as a whole, unoccupied patches are colonised more quickly due to increased dispersal from the occupied patches.

\textcite{Hunt2009} extended this model to multi-species communities by adding competition to the extinction term:

\begin{equation}
	\frac{\partial p_i}{\partial t} = c_i p_i \left( h-p_i \right) - p_i \left( \sum_{j \in S \setminus \left\{i \right\} }{m_{ij}p_j} + m_i \right) \label{eq:hunt}
\end{equation}
where the subscript $i$ indicates a focal species, the subscript $j$ a competitor, and $S \setminus \left\{i \right\}$ is the set of species in the local community excluding the focal species $i$.
Here, the extinction rate is now broken into two terms.
The first a species-specific intrinsic extinction rate $m_i$, representing, e.g., stochastic extinctions that are unrelated to other species in the local community.
The $m_{ij}$ term is the effect of competition between $i$ and $j$ on the extinction rate of $i$ (multiplied by $p_j$ because competition only occurs when species $j$ is also present).
For future clarity of notation, we assume all parameters are specific to a target species $i$ and thus omit the subscript.

An important aspect of metacommunity dynamics is dispersal, which in the classic model is incorporated into the colonisation term using the prevalence $p_i$.
In rivers, hydrological connections among habitat patches facilitates passive dispersal in the water (cite).
However, many organisms are capable of active dispersal, either overland or upstream along the streambed, and even organisms that cannot move themselves can be carried upstream by various properties (cite).
For simplicity, we refer to all of these forms of dispersal as “active” dispersal.
Because these two dispersal mechanisms are likely to occur at quite different rates, and because the rates will likely vary among different types of organisms, we break the dispersal portion of the colonisation term here into active ($\alpha$) and passive ($\beta$) components, with the passive component additionally weighted by the discharge $Q$:

\begin{equation}
	\frac{\partial p}{\partial t} = c p(\alpha + \beta Q) \left( h-p \right) - p \left( \sum_{j \in S \setminus \left\{i \right\} }{m_{j}p_j} + m \right)
	\label{eq:metacom}
\end{equation}

In order to produce a useful model where the meta-community interacts with the environment, it is necessary to further extend this model to incorporate local (and potentially dynamic) non-biological conditions in specific patches.
More recent theoretical \autocite{Holt2000,Holt2005} and empirical \autocite{Talluto2017} work has extended the single-species Levins model (eqn. \ref{eq:levins}) by fitting the colonisation and extinction rates as functions of local climatic conditions.
The result is a dynamic range model with long-term occupancy driven by the balance of local colonisation and extinction \autocite{Talluto2017}.
We combine this approach with the multi-species Hunt model (eqn. \ref{eq:hunt}) by redefining the $c$ and $m$ terms to be functions of the quality $q_x$ of a focal patch $x$:

\begin{equation}
\begin{split}
	c_{x} &= f(q_{x}) \label{eq:talluto} \\
	m_{x} &= g(q_{x})
\end{split}
\end{equation}


The shape of these two functions is flexible; for example \textcite{Talluto2017} defined them as quadratic functions of local climate conditions.
Our more general approach here can incorporate climatic conditions, local habitat, and dynamical resource concentrations into the $q$ term.
We explore these possibilities further in the case studies.
For now, we consider the case where $q_x = R_x$, the concentration of an essential resource.
We can use a simple reaction-transport model (cite) to describe the fluctuations of this resource concentration in time (the subscript $x$ has been omitted for clarity).

\begin{equation}
	\frac{\partial R}{\partial t} = \sum_{i \in S}{\rho_i(R)} -\frac{QR - \sum Q_u R_u}{A l} 
	\label{eq:rxn_transport}
\end{equation}
The right-hand term here gives loss due to advective transport, where $Q$ is the discharge of the focal patch (in volumetric units per unit time; we use m$^3$ s$^{-1}$ throughout), $Q_u$ and $R_u$ are discharge and resource concentrations of upstream patches (including lateral input, if relevant), $A$ is the cross-sectional area (m$^2$), and $l$ is the stream length of the patch (m).
The left-hand term is the reaction component.
We consider here only reaction due to biological activity, and postulate that the net reactive change in a patch is the sum of a set of resource use functions, $\rho_i(R)$, of all species $S$ in the local community. 
Each $\rho_i$ function describes the impact of a species on the resource; the forms of these functions will depend on which resources are being modelled.
We provide examples of possible functions in the case studies.


\subsection{Simulations}

For analysis, we discretised the model in space and time.
We thus consider a series of habitat patches representing stream reaches.
Each reach is characterised by the state variable $R$, the resource concentration, and by the community vector $\mathbf{C}$, which gives the absence (denoted by zeroes) or presence (denoted by ones) of all possible species in the community.
For a time interval $\Delta t$ we can then derive the probability of observing colonisations and extinctions for a species $i$ from eq. \ref{eq:metacom}:

\begin{equation}
\begin{split}
	\mathrm{pr}\left( \mathbf{C}_{i, t+\Delta t} = 1 \mid \mathbf{C}_{i, t} = 0\right) &= 
			1 - \mathrm{e}^{-c_i p_i(\alpha_i + \beta_iQ) \Delta t} \\
	\mathrm{pr}\left( \mathbf{C}_{i, t+\Delta t} = 0 \mid \mathbf{C}_{i, t} = 1\right) &= 
			1 - \mathrm{e}^{-\left( \sum{m_{ij}\mathbf{C}_{j, t}} + m_i \right)\Delta t}
	\label{eq:ceprob}
\end{split}
\end{equation}
Here, the prevalence term $p_i$ indicates local prevalence, i.e., the number of occupied patches within dispersal range.
For our purposes, we consider only the case of nearest neighbour dispersal, so $p_i$ is the sum of the occupancy states of patches immediately up- and downstream of the focal patch; longer distance dispersal can be incorporated via the regional species pool (see §\ref{ss:initial-boundary}).

Changing resource concentration can be computed for each patch using a variety of numerical integration techniques.
In all of our examples, we consider a sufficiently short time step (i.e., less than 30 min) that Euler integration provides sufficient precision (cite).

\subsection{Initial \& Boundary Conditions} \label{ss:initial-boundary}

Because of the directional nature of the movement of resources (and, to a lesser extent, organisms) in the model, it is necessary to provide external input representing, e.g., the input of resources and organisms from the terrestrial matrix or from groundwater input.
For each habitat patch in the model, we implemented a virtual upstream patch representing this input.
The resource concentrations and community composition of these patches are constant.
Discharge from virtual patches is computed simply as the growth in discharge from one reach to the next moving downstream, according to the growth in catchment area and defined by hydraulic scaling relationships (cite).
The actual resource concentration and community composition of the virtual patches can vary depending on modelling needs; for example, community composition could be uniform and contain all possible species to represent a classical “regional species pool” from metacommunity modelling, or resource concentrations could vary among headwaters or from upstream to downstream to represent land use gradients.


\subsection{Case studies}

\subsubsection{Simulated Algal-N:P meta-ecosystem}
As an initial motivating example, we consider a simulated algal metacommunity, where species are differentiated along a single niche dimension.
We use the nitrogen to phosphorous (N:P) ratio.
This is a convenient example, as the N:P ratio is known to be quite important in algal communities (), and it can be easily modelled as a single scalar, thereby adding simplicity to the resource component of the model.	

\begin{itemize}
	\item niche descriptions/distribution (i.e., gaussian curves)
	\item strength of competition (overlap of curves)
	\item resource use function (effect of algae on resource)---proportional to height of gaussian curve at the existing N:P ratio, but what is the actual effect on N:P?
	\item communities tested
	\item landscapes tested 
\end{itemize}

\subsubsection{Adding a habitat/niche dimension? (for Gabriel)}

\subsubsection{Vjosa case study?}

\section{Results}

\section{Discussion}

\begin{itemize}
	\item limitation: we don't consider abundance, only presence-absence
\end{itemize}

\printbibliography
\end{document}
